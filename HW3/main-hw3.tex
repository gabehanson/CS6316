
\documentclass[10pt]{article}

\usepackage{graphicx}
\usepackage{fancybox}
\usepackage{epsfig}
\usepackage{float}
\usepackage{amsmath,amsfonts,amssymb}
\usepackage{subfigure}
\usepackage{multirow}
\usepackage{bm}
\usepackage[]{algorithm2e}
\usepackage{color}
\usepackage{hyperref}

\urlstyle{same}


\usepackage{amsmath,amsfonts,amssymb,bm}
\usepackage{pifont} % http://ctan.org/pkg/pifont
\usepackage{graphicx,subfigure,epsfig,fancybox} % figures & graph
\usepackage{float}
\usepackage{color} % color
\usepackage[]{algorithm2e} %
\usepackage{multirow}

\newcommand{\mycite}[1]{\small{(#1)}}
\newcommand{\reviseit}[1]{\textcolor{cyan}{[#1 - DOUBLECHECK]}}
\renewcommand{\vec}[1]{\boldsymbol{#1}}
\newcommand{\eq}[1]{\left\langle #1 \right\rangle}
\newcommand{\mat}[1]{\mathbf{#1}}
\newcommand{\trans}[1]{#1^{\textsf{T}}}
\newcommand{\num}[1]{\text{\#}(#1)}
\newcommand{\inv}[1]{(#1)^{-1}}
\newcommand{\Inv}[1]{\left(#1\right)^{-1}}
\newcommand{\set}[1]{\mathcal{#1}}
\newcommand{\diag}[1]{\text{diag}\left(#1\right)}
\newcommand{\annotation}[1]{\textsc{#1}}
\newcommand{\annot}[1]{\annotation{#1}}
\newcommand{\tagword}[2]{\example{#1}/\annot{#2}}
\newcommand{\phon}[1]{\textsf{#1}}
% \newcommand{\argmax}[1]{\text{arg}\max_{#1}}
% \newcommand{\argmin}[1]{\text{arg}\min_{#1}}
\DeclareMathOperator*{\argmin}{argmin}
\DeclareMathOperator*{\argmax}{argmax}
\newcommand{\expect}[1]{{E}\left[{#1}\right]}
\newcommand{\dd}[2]{\frac{\partial #1}{\partial #2}}
\renewcommand{\symbol}[1]{$\langle$#1$\rangle$}
\newcommand{\sentstart}[0]{\symbol{s}}
\newcommand{\sentend}[0]{\symbol{/s}}
\newcommand{\tuple}[1]{\langle #1 \rangle}
\newcommand{\cue}[1]{[\textcolor{blue}{#1}]}
\newcommand{\question}[1]{\textcolor{red}{#1}}
\newcommand{\fstransx}[2]{\xrightarrow[\example{#2}]{\example{#1}}}
\newcommand{\fstrans}[2]{\xrightarrow[{#2}]{#1}}
\newcommand{\gloss}[1]{{\fontfamily{phv}\selectfont #1}}
\newcommand{\newadd}[1]{\textcolor{magenta}{#1}}

\newcommand{\from}[0]{\leftarrow}
\newcommand{\reals}[0]{\mathbb{R}}
\newcommand{\normal}[1]{\mathcal{N}\left(#1\right)}
\newcommand{\cpic}[2][.95\textwidth]{\begin{center}\includegraphics[width=#1]{#2}\end{center}}
% \newcommand{\etal}[0]{\textit{et al.}~}
\newcommand{\bigo}[1]{\mathcal{O}(#1)}
\newcommand{\Pmult}[0]{P_{\text{multinomial}}}
\newcommand{\Pcat}[0]{P_{\text{categorical}}}
\newcommand{\vw}[0]{\vec{w}}
\newcommand{\vwt}[0]{\trans{\vec{w}}}
\newcommand{\vd}[0]{\vec{d}}
\newcommand{\vf}[0]{\vec{f}}
\newcommand{\vx}[0]{\vec{x}}
\newcommand{\vz}[0]{\vec{z}}
\newcommand{\vy}[0]{\vec{y}}
\newcommand{\vs}[0]{\vec{s}}
\newcommand{\vu}[0]{\vec{u}}
\newcommand{\vth}[0]{\vec{\theta}}
\newcommand{\vtht}[0]{\trans{\vec{\theta}}}
\newcommand{\vph}[0]{\vec{\phi}}
\newcommand{\jm}[0]{J\&M}
\newcommand{\dkl}[0]{\mathcal{D}_{KL}}
\newcommand*{\defeq}{\mathrel{\vcenter{\baselineskip0.5ex \lineskiplimit0pt\hbox{\scriptsize.}\hbox{\scriptsize.}}}=}
% \newcommand{\refeq}[1]{(\ref{#1})}

\newenvironment{exampleenv}{\rmfamily}{\normalfont}


\usepackage{adjustbox}
\usepackage{array}
\usepackage{booktabs}

\newcolumntype{R}[2]{%
    >{\adjustbox{angle=#1,lap=\width-(#2)}\bgroup}%
    l%
    <{\egroup}%
}
\newcommand*\rot{\multicolumn{1}{R{45}{0.4em}}}


\newcommand{\todo}[1]{[\textcolor{red}{\tt TODO: #1}]}
\newcommand{\seqlabel}[1]{\vec{y}^{(#1)}} % sequence labels
\newcommand{\dotp}[2]{\langle #1, #2\rangle}
\newcommand{\pd}[2]{\frac{\partial #1}{\partial #2}}
\newcommand{\cmark}{\ding{51}} % correct mark
\newcommand{\xmark}{\ding{55}} % cross mark



\parskip 2.0mm
\topmargin  -2.0cm
\textwidth 16cm
\textheight 22.5cm
\topmargin -1.5cm
\oddsidemargin 0mm

\title{Homework 03: Neural Networks and Generative Models}
\author{Gabriel Hanson}
\date{Due on Dec 1, 2022 11:59 PM}

% ***************************************************
\begin{document} 
\maketitle
% \tableofcontents

\begin{enumerate}
\item \textbf{The backpropagation algorithm} (12 points) introduces the fundamental algorithm for training neural networks, called the backpropagation algorithm. Essentially, the calculation of backpropagation is an application of the chain rule in calculus. For example, if function $h(x)$ is composed by two simple function $g_1(\cdot)$ and $g_2(\cdot)$ as
  \begin{equation}
    h(x) = g_1(g_2(x))
  \end{equation}
  then, the gradient of $h(x)$ with respect to $x$ is
  \begin{equation}
    \frac{d h}{d x} = \frac{d g_1(g_2(x))}{d g_2(x)}\frac{d g_2(x)}{d x}.
  \end{equation}

  In this problem, let us consider a special neural network architecture, called \emph{recurrent neural networks} (RNNs).
  RNNs are often used to model sequential data, such as word sequences in NLP or time series data.
  Along a sequence, a RNN is often specified by a recursive function as 
  \begin{equation}
    h_t = \sigma(w_1x_t + w_2h_{t-1})
  \end{equation}
  where $\sigma(\cdot)$ is the Sigmoid function, $w_1$ and $w_2$ are the parameters of this RNN.  $h_t$ is called the \emph{hidden state} at time stamp $t$, which is a function of the current input $x_t$ and the previous hidden state $h_{t-1}$.
  As a simple starting point, we usually set $h_0=0$.
  \begin{itemize}
  \item[(a)] (2 points) Assume that we have a sequential data with total time stamp $T=3$, please write down the explicit expression of $h_1$, $h_2$, and $h_3$. In other words, for each $h_t$, please only use the composition of the Sigmoid functions together with $\{x_1,\dots,x_t\}$ and $\{w_1,w_2\}$. \\ \\ 
  \hspace{\parindent}\textit{\textbf{Solution:}}
$$  h_1 = \sigma(w_1x_1 + w_2h_0) =  \sigma(w_1 x_1 ) $$ 
$$  h_2 = \sigma(w_1x_2 + w_2h_1) =  \sigma(w_1 x_2 + w_2 \sigma(w_1 x_1))) $$ 
$$ h_3 =  \sigma(w_1x_3 + w_2h_2) =  \sigma(w_1 x_3 + w_2 \sigma(w_1 x_2  + w_2 \sigma(w_1 x_1))) $$ 
  
  
  \item[(b)] (2 points) Based on the explicit expressions of $\{h_t\}_{t=1}^{3}$ in (a), please write down the derivative of $h_t$ with respect to $w_1$, $$\frac{d h_t}{d w_1}$$ where $t=1,2,3$. \par
 \hspace{\parindent}\textit{\textbf{Solution:}}
$$  \frac{\partial h_1}{\partial w_1} =   \frac{\partial }{\partial w_1} (\sigma(w_1 x_1) )=  \sigma(w_1 x_1)( 1 - \sigma(w_1 x_1)) (x_1) $$ 
$$  \frac{\partial h_2}{\partial w_1} =   \frac{\partial }{\partial w_1}[ \sigma(w_1 x_2 + w_2 \sigma(w_1 x_1)))] = \sigma(w_1 x_2 + w_2 \sigma(w_1 x_1)) (1 - \sigma(w_1 x_2 + w_2  \sigma( w_1 x_1))) $$
$$ (x_2 + w_2\sigma(w_1 x_1)(1-\sigma(w_1 x_1)) (x_1) $$
$$  \frac{\partial h_3}{\partial w_1} =  \frac{\partial }{\partial w_1} [ \sigma(w_1 x_3 + w_2 \sigma(w_1 x_2  + w_2 \sigma(w_1 x_1)))] = $$ 
$$ [\sigma(w_1 x_3 + w_2 \sigma(w_1 x_2 + w_2 \sigma (w_1 x_1 )))][ 1 - \sigma( w_1 x_3 + w_2 \sigma(w_1 x_2 + w_2 \sigma( w_1 x_1)))]$$
$$ [x_3 + w_2 \sigma (w_1 x_2 + w_2 \sigma(w_1 x_1))][1- \sigma(w_1x_2 + w_2 \sigma(w_1 x_1))] [ x_2 + w_2 (\sigma(w_1 x_1) (1 - \sigma(w_1 x_1))) ] [x_1 ] $$

  \item[(c)] (2 points) Without specify the value of $t$, please write down the \emph{general} form of $$\frac{d h_t}{d w_1},$$ where $t$ can be any positive integer. \par
  \hspace{\parindent}\textit{\textbf{Solution:}}
$$  \frac{\partial h_t}{\partial w_1} = \sigma(w_1 x_t + w_2 (h_{t-1}))( 1 - \sigma( w_1 x_t + w_2 (h_{t-1})))(x_t + w_2 \frac{\partial h_{t-1}}{\partial w_1})$$ \par
  \item[(d)] (2 points) A typical way of using RNNs is to take the hidden states for some prediction tasks. For simplicity, we assume that for the following classification problem, we only use the last hidden state $h_T$. Then, the corresponding classifier is defined as
    \begin{equation}
      p(y\mid \vx) = \sigma(y (w_o h_T + b))
    \end{equation}
    where $w_o$ and $b$ are parameters associated with this classifier, and $y\in\{-1,+1\}$.
    
    In the case of stochastic gradient descent, given one training example $(\vx, y)$, if we use the cross entropy as the loss function $L$, please compute the gradient of $L$ with respect to $w_1$.\par
 \hspace{\parindent}\textit{\textbf{Solution:}}

$$ \frac{\partial L}{\partial w_1} = \frac{\partial L}{\partial p} * \frac{\partial p}{\partial h_T} * \frac{\partial h_T}{\partial w_1} $$
$$  \frac{\partial L}{\partial p} =  \frac{\partial }{\partial p}( -y \log(p) - (1 - y )\log(1 - p)) =  \frac{ - y }{ p } - \frac{(1- y)}{(1 -p)}$$
$$  \frac{\partial p}{\partial h_T} =  \frac{\partial }{\partial h_T} (\sigma(y(w_0 h_T +b))) = \sigma(y(w_0 h_T +b))*(1 - \sigma(y(w_0 h_T +b ))) * w_0 $$ 
\\ and as derived  above $$  \frac{\partial h_t}{\partial w_1} = \sigma(w_1 x_t + w_2(h_t - 1))(1 - \sigma(w_1 x_t + w_2 (h_{t-1})))(x_t + w_2 \frac{\partial h_{t-1}}{\partial w_1}) $$
\\ plugging back in and letting $$h_{t-1} = 0$$  $$ \frac{\partial L}{\partial w_1} =  [\frac{ - y }{ p } - \frac{(1- y)}{(1 -p)}]*[ \sigma(y(w_0 h_T +b))*(1 - \sigma(y(w_0 h_T +b ))) * w_0]*$$
$$ [ \sigma(w_1 x_t )(1 - \sigma(w_1 x_t ))(x_t ) ] $$
substituting $$ p =  \sigma(y(w_0 h_T +b)) $$
\\
\\ we get
$$ =  [\frac{ - y }{ \sigma(y(w_0 h_T +b)) } - \frac{(1- y)}{(1 - \sigma(y(w_0 h_T +b)))}]*[ \sigma(y(w_0 h_T +b))*(1 - \sigma(y(w_0 h_T +b ))) *[ w_0]* [ \sigma(w_1 x_t )(1 - \sigma(w_1 x_t ))(x_t ) ] $$
$$  =  [\frac{ (- y)( \sigma(y(w_0 h_T +b))*(1 - \sigma(y(w_0 h_T +b ))) ) }{ \sigma(y) } - \frac{(1- y) \sigma(y(w_0 h_T +b))*(1 - \sigma(y(w_0 h_T +b ))) }{(1 -\sigma(y))}]*[ w_0]* $$ 
$$ \sigma(w_1 x_t )(1 - \sigma(w_1 x_t ))(x_t )  $$
 I will write $$ y(w_0 h_T +b)) =  y $$
then our gradient becomes $$ \frac{\partial L}{\partial w_1} = [(-y)(1-\sigma(y)) - (1-y)(\sigma(y))]* w_0* \sigma(w_1 x_t )(1 - \sigma(w_1 x_t ))(x_t ) $$

  \item[(e)] (2 points) Now, let us consider the Sigmoid function $\sigma(r)$, based on its definition
    \begin{equation}
      \sigma(r) = \frac{1}{1 + \exp(-r)}
    \end{equation}
    please show that, for $r\in\mathbb{R}$
    \begin{equation}
      0 < \frac{d \sigma(r)}{d r} < 0.5
    \end{equation} \par
     \hspace{\parindent}\textit{\textbf{Solution:}}
see attached plots and code entitled hw3Q1.ipynb for plots. As you can see from plot, the derivative of the sigmoid function reaches its max value at r = 0, but never rises above 0.25. So we can say: 
$$      0 < \frac{d \sigma(r)}{d r} < 0.5 $$

  \item[(f)] (2 points) Combine (d) and (e), please explain the \emph{vanishing gradient} problem in learning RNNs with a large $T$ (say, $T\geq 50$). Your explanation needs to be as \emph{specific} as possible. In other words, the explanation should be established on the results from (d) and (e).
  \end{itemize} \par
       \hspace{\parindent}\textit{\textbf{Solution:}}
I believe I have made a mistake in part d at some point. But, alas I will try my hand at answering this anyways. The vanishing gradient occurs during the backpropogation step when updating neuron weights. The recurrent nature of the network means that the cost function of the neurons in the deeper parts of the network will influence the weights of the shallower neurons. Because the influence is multiplicative, it is very sensitive to the weights being back propogated. As we can see in part e, there is only a small range for values of r where the the gradient of sigmoid is nonzero. If you go much beyond r = 0.5, that derivative will go to zero and this will then propogate back through the network causing an exploding or vanishing gradient. 

\item \textbf{Model selection} (8 points) Please create an iPython notebook file for answering the following questions. For this part, 
  \begin{enumerate}
  \item for each code block, mark it clearly which question it is for
  \item please make sure the code is execuatable with any modification, for example, importing necessary packages
  \item \textbf{report the accuracy numbers in the pdf file}, and our TAs will re-run the code and compare with the numbers reported in the pdf file
  \end{enumerate}
  On piazza, along with this pdf file, there are five files in \texttt{data.tgz}
  \begin{itemize}
  \item trnX-3d.csv, trnY-3d.csv: the input/output files for the training set
  \item devX-3d.csv, devY-3d.csv: the input/output files for the development set
  \item tstX-3d.csv: the input file for the test set
  \end{itemize}
  Questions:
  \begin{itemize}
  \item[(a)] (3 points) Please use the {MLPClassifier} from sklearn with the default hyper-parameter setting for training.
  
  (\small \url{https://scikit-learn.org/stable/modules/generated/sklearn.neural_network.MLPClassifier.html}) 
  
  For your convenience, you may want to use the {accuracy function}  provided by sklearn.
  
  (\small \url{https://scikit-learn.org/stable/modules/generated/sklearn.metrics.accuracy_score.html})
  
 

    Report the accuracy on both the training and development sets in the pdf file.
  \item[(b)] (5 points) MLPClassifier provides about 20 hyper-parameters for tuning. In this question, we will limit to just a few of them.
    \begin{itemize}
    \item hidden\_layer\_sizes: note that, this parameter will let you specify the number of layers and the hidden units in each layer
    \item activation
    \item solver: please only use SGD or Adam for hyper-parameter tuning
    \item learning\_rate\_init
    \item batch\_size
    \end{itemize}
    For these five hyper-parameters, you can choose any values for them as you wish (except the specified constraint for ``solver'').
    The goal of hyper-parameter tuning is to find the best results on the development set.
    For your convenience, you may want to use the GridSearchCV function in the sklearn.

    Please report the following information in the PDF file
    \begin{itemize}
    \item the best development accuracy
    \item the corresponding hyper-parameters that produced them
    \item the difference between the best accuracy and the accuracy produced by the default hyper-parameters
    \end{itemize}
  \end{itemize}
 \hspace{\parindent}\textit{\textbf{Solution:}}
see hw3.ipynb file
\end{enumerate}

% *******************END OF DOCUMENT*****************
\end{document}

